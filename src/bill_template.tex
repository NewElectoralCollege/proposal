\documentclass{article}
\usepackage{bills}
\title{The New Electoral College proposed legislation - States}
\author{David Boers}

\begin{document}
    \type{act}
    \subject{To ammend the method by which the State of [state] appoints Electors for President and Vice President.}
    \formula{state}
    \namedsection{1}{To prescribe the method of appointing electors}
    \begin{enumerate}
        \item The Electors for President and Vice President from the State of [state] shall be appointed by Popular Election.
        \item In the Election for Presidential Electors, the entire State shall constitute a single Electoral District.
        \item The Voting System for the Election for Presidential Electors shall be Party-list proportional representation.
    \end{enumerate}
    \namedsection{2}{Electoral Lists}
    \begin{enumerate}
        \item In the Popular Election for Electors, voters shall cast their vote for a List of Elector Candidates.
        \item The number of Elector Candidates on an Electoral List shall not be less than the number of Electors to which the State is entitled.
        \item An Electoral List may represent---
        \begin{enumerate}
            \item An organization designated as a political party by the electoral officer of the State;
            \item A campaign committee for an independent candidate for President; or
            \item An alliance of political parties.
        \end{enumerate}
        \item Electoral Lists shall be submitted to the electoral officer of the State not later than two months prior to polling date for Electors.
        \item The electoral officer of a State shall scrutinize the submitted Electoral Lists, and may strike Elector Candidates from them for any of the following reasons---
        \begin{enumerate}
            \item being listed on multiple Lists;
            \item the wish of the Elector Candidate;
            \item the wish of the entity that submitted the List;
            \item the identity of the intended Elector Candidate being non-discernible;
            \item the Elector Candidate is not eligible to be an Elector under Article 2, Section 1, Paragraph 2 of the Constitution of the United States; or
            \item non-conformance with State law regarding Elector Candidates.
        \end{enumerate}
        \item If Elector Candidates being struck from an Electoral List results in the number of Elector Candidates on the List falling below the number specified in subsection (b), the entity that submitted the Electoral List shall be entitled to submit replacement Elector Candidates such that the number of Elector Candidates on the List is valid under subsection (b).
        \item Electoral Lists shall be publicly released not later than one month prior to the polling date for Electors set by the respective State, and they shall also be available for public viewing at all polling stations.
        \item Each Elector Candidate on a List shall be given a rank by the submitting entity; no two Elector Candidates are to be awarded the same rank on the List.
        \item A List shall declare the candidates it has endorsed for President and Vice President, and this shall be listed on the ballot; but a List may neglect to endorse a candidate for either office.
    \end{enumerate}
    \namedsection{3}{Allocation of Electors to Candidates}
    \begin{enumerate}
        \item The number of votes cast for each candidate shall be the sum of the number of votes cast for all Lists that nominated that candidate for President.
        \item Electors shall first be allocated to candidates in the following manner---
        \begin{enumerate}
            \item The total number of valid votes cast for Lists divided by the total number of Electors to which the State is entitled, rounded down to the nearest integer, shall constitute an electoral quota.
            \item Candidates shall be awarded an initial number of Electors equal to the number of times that the number of votes cast for them can be equally divided by the quota.
            \item After the calculation described in subsection (b)(2) is performed, the number of votes for each candidate shall be diminished by the product of the quota and the number of Electors awarded to that candidate in subsection (b)(2).
            \item Electors that can not be awarded by the method described in subsection (b)(2) shall be awarded to candidates in descending order of their vote total after the alteration described in subsection (a)(4).
        \end{enumerate}
        \item Should any tie occur between two or more candidates such that it can not be determined which
        shall be awarded an Elector, the candidate that was registered first shall be awarded the Elector.
    \end{enumerate}
    \namedsection{4}{Allocation of Electors to Lists}
    \begin{enumerate}
        \item After the allocation of Electors to candidates, the Electors won by each candidate shall be allocated between the Lists that nominated that candidate in the following manner---
        \begin{enumerate}
            \item If a single List nominated the candidate, all of the candidate's Electors shall be awarded to that List; otherwise
            \item Electors shall be allocated between the Lists using the same calculation method outlined in subsection (3)(b), but where the quota is the number of votes cast for the candidate, divided by the number of Electors won by the candidate, rounded down to the nearest integer.
        \end{enumerate}
        \item Should any tie occur between two or more Lists such that it can not be determined which
        shall be awarded an Elector, the List that was registered first shall be awarded the Elector.
    \end{enumerate}
    \namedsection{5}{Selection of Electors}
    \begin{enumerate}
        \item Electors shall be chosen from each List in the number that the List is entitled in the order of the ranks of the Elector Candidates.
        \item Should an Elector, prior to casting their vote, resign, be found ineligible, or otherwise incapacitated, they shall be replaced with the next Elector Candidate on their respective List.
        \item The electoral officer shall certify the election of the Electors, and present the certification to the governor not less than two weeks succeeding the election, and the said certificate shall contain---
        \begin{enumerate}
            \item The Lists of Elector Candidates;
            \item The number of votes cast for each List;
            \item The number of votes cast for each candidate;
            \item The electoral quota;
            \item The number of Electors allocated to each candidate under subsection (3)(b)(2);
            \item The number of Electors allocated to each candidate under subsection (3)(b)(4);
            \item The number of Electors allocated to each List under subsection (3)(b)(2);
            \item The number of Electors allocated to each List under subsection (3)(b)(4);
            \item The elected Elector Candidates.
        \end{enumerate}
        \item If the governor determines that the certification is in order and lawful, they shall sign it, thereupon appointing the Electors.
        \begin{enumerate}
            \item If the governor determines that there is a mathematical error in the certification, or that it is not in order and lawful, they may correct the certification, but if the electoral officer gives successfull legal challenge to the correction, the correction is reversed.
            \item If the governor takes no action on the certification before a week proceeding the meeting of the Electors, it shall be considered of effect as if the governor had signed it. 
        \end{enumerate}
    \end{enumerate}
    \namedsection{6}{Ballots}
    \begin{enumerate}
        \item Electoral Lists shall be placed on the ballot in a Popular Election for Electors in accordance with state law.
        \item If an entity described in subsection (2)(c)(1) submits multiple lists, they shall declare a special designation for each list, which must be listed on the ballot in addition to the name of the entity.
        \item If there are multiple Lists representing an entity or entities described in subsection (2)(c)(2), they shall be given a letter designation representing the order in which the Lists were registered.
        \item Lists representing an entity described in subsection (2)(c)(3), they shall be designated in whatever manner the entity declares, but the name shall end in the word "List". 
    \end{enumerate}
\end{document}